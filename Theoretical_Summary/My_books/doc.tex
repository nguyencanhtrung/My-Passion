%%%%%%%%%%%%%%%%%%%%%%%%%%%%%%%%%%%%%
% Inlucudings:                      %
%%%%%%%%%%%%%%%%%%%%%%%%%%%%%%%%%{{{%
\documentclass[11pt,english,a4paper,chapterprefix]{scrartcl}
% Bring old fonts back
\makeatletter
\DeclareOldFontCommand{\rm}{\normalfont\rmfamily}{\mathrm}
\DeclareOldFontCommand{\sf}{\normalfont\sffamily}{\mathsf}
\DeclareOldFontCommand{\tt}{\normalfont\ttfamily}{\mathtt}
\DeclareOldFontCommand{\bf}{\normalfont\bfseries}{\mathbf}
\DeclareOldFontCommand{\it}{\normalfont\itshape}{\mathit}
\DeclareOldFontCommand{\sl}{\normalfont\slshape}{\@nomath\sl}
\DeclareOldFontCommand{\sc}{\normalfont\scshape}{\@nomath\sc}
\makeatother

%\usepackage[T1]{fontenc}
\usepackage[bigcaptions]{listing}
%\usepackage[latin1]{inputenc}
\usepackage[small,bf,hang]{caption}
\usepackage[english]{babel}
%\usepackage{epsfig}
\usepackage{wrapfig}
%\usepackage{caption}
\usepackage{psfrag}
\usepackage[rflt]{floatflt}
\usepackage[usenames]{color}
\usepackage{graphicx}
\emergencystretch = 10pt
\usepackage{amsmath}
\usepackage{amssymb}
\usepackage{setspace}
%\usepackage{calc}
\usepackage{tocloft}
\usepackage{listing}
\usepackage{listings}
\usepackage{trsym}
\usepackage{trfsigns}
\usepackage{minted}
\usepackage{multirow}
\usepackage{fancyhdr}
\usepackage{nomencl}
\usepackage{todonotes}
\usepackage{float}
\usepackage{subfig}
\usepackage{url}
\usepackage{hyperref}

\usepackage[utf8]{inputenc}
\usepackage[tcvn]{vietnam}
%\usepackage{listings}
%\input{subsections.sty}
\setcounter{secnumdepth}{5}
\setcounter{tocdepth}{5} 
\numberwithin{equation}{section}
\numberwithin{figure}{section}

%matrix environment redef
\makeatletter
\renewcommand*\env@matrix[1][*\c@MaxMatrixCols c]{%
	\hskip -\arraycolsep
	\let\@ifnextchar\new@ifnextchar
	\array{#1}}

\renewenvironment{bmatrix}
{{\ifnum`}=0 \fi\left[\env@matrix}
{\endmatrix\right]\ifnum`{=0 \fi}}
\makeatother


%NOTE: added from paper.texexamples-master (GIT) to create nice tables
\usepackage{booktabs} % f�r sch�nere Tabellen mit unterschiedlich dicken Linien
\usepackage{color} % f�r farben
\usepackage{colortbl} % f�r farbige tabellen
\usepackage{rotating} % Tabellen drehen
\usepackage{multirow}	% mehrzeilige Tabelleneintr�ge
\usepackage{multicol} % mehrspaltige Tabelleneintr�ge
\usepackage{url} % to allow urls
\usepackage{tabularx}
\usepackage{threeparttable} % tables with footnotes
\usepackage{stfloats} % allow double column table at the bottom
\definecolor{mygreen}{RGB}{28,172,0} % color values Red, Green, Blue
\definecolor{mylilas}{RGB}{170,55,241}

%% math packages, Songyin Tang
\usepackage{amsmath}	
\usepackage{amsfonts, amssymb}	
\usepackage{algorithm}  
%\usepackage{algorithmic}  
\usepackage{algorithmicx}  
\usepackage{algcompatible}
\renewcommand{\algorithmicrequire}{\textbf{Input:}}
\renewcommand{\algorithmicensure}{\textbf{Output:}}
\usepackage{graphicx} 

%%%%%%%%%%%%%%%%%%%%%%%%%%%%%%%%%}}}%
% New Commands and Configurations:  %
%%%%%%%%%%%%%%%%%%%%%%%%%%%%%%%%%{{{%
%\setkomafont{section}{\Large\rmfamily}
%\setkomafont{subsection}{\large\rmfamily}
%\setkomafont{subsubsection}{\normalsize\rmfamily}
\setkomafont{paragraph}{\footnotesize}
\numberwithin{table}{section}
\numberwithin{listing}{section}
\setlength\textheight{24cm}
\definecolor{orange}{rgb}{1 , 0.5 , 0}
\definecolor{blue}{rgb}{0, 0 , 1}
\definecolor{green}{rgb}{0, 1 ,0}
\newcommand{\cb}{\textcolor{blue}}
\newcommand{\subsubsubsection}{\paragraph}
\newcommand{\subsubsubsubsection}{\subparagraph}
\clubpenalty = 10000
\widowpenalty = 10000
\displaywidowpenalty = 10000
\parindent0pt % No Indent
\makenomenclature
% Document Head

\begin{document}
	%\restylefloat{figure}
	\pagestyle{fancy}
	\rhead{} 
	
	\definecolor{light-gray}{gray}{0.95}
	
	\newminted{cpp}{bgcolor=light-gray, fontsize=\scriptsize}
	\newminted{tcl}{bgcolor=light-gray, fontsize=\scriptsize}
	\newminted{sh}{bgcolor=light-gray, fontsize=\scriptsize}
	\newminted{basemake}{bgcolor=light-gray, fontsize=\scriptsize}
	
	%%%%%%%%%%%%%%%%%%%%%%%%%%%%%%%%%}}}%
	% fancy nomenclautur:
	%%%%%%%%%%%%%%%%%%%%%%%%%%%%%%%%%{{{%
	%\setlength{\nomlabelwidth}{.20\hsize}
	%\renewcommand{\nomlabel}[1]{#1 \dotfill}
	
	%<*sample05>
	\def\@@@nomenclature[#1]#2#3{%
		\def\@tempa{#2}\def\@tempb{#3}%
		\protected@write\@nomenclaturefile{}%
		{\string\nomenclatureentry{#1\nom@verb\@tempa @[{\nom@verb\@tempa}]%
				|nompageref{\begingroup\nom@verb\@tempb\protect\nomeqref{\theequation}}}%
			{\thepage}}%
		\endgroup
		\@esphack}
	
	\def\pagedeclaration#1{\dotfill\nobreakspace ~#1}
	%\def\nomentryend{.}
	\def\nomlabel#1{\textbf{#1}\hfil}
	\makeatletter 
	\renewcommand*\dotfill{\leavevmode% 
		\leaders\hbox{$\m@th 
			\mkern \@dotsep mu\hbox{.}\mkern \@dotsep 
			mu$}\hfill\kern\z@} 
	\makeatother
	%%%%%%%%%%%%%%%%%%%%%%%%%%%%%%%%%}}}%
	% Abbr Commands!
	%%%%%%%%%%%%%%%%%%%%%%%%%%%%%%%%%{{{%
	\newcommand{\abbr}[2]{\textit{#2} (#1)\nomenclature{#1}{#2 \nomrefpage}}
	\newcommand{\shortabbr}[2]{\nomenclature{#1}{#2 \nomrefpage}}
	\newcommand{\revabbr}[2]{#1 (\textit{#2})\nomenclature{#1}{#2 \nomrefpage}}
	
	%%%%%%%%%%%%%%%%%%%%%%%%%%%%%%%%%}}}%
	% Titlepage                         %
	%%%%%%%%%%%%%%%%%%%%%%%%%%%%%%%%%{{{%
	\input{"0.titlepage.tex"}
	\newpage
	%%%%%%%%%%%%%%%%%%%%%%%%%%%%%%%%%}}}%
	% Table of Contents                 %
	%%%%%%%%%%%%%%%%%%%%%%%%%%%%%%%%%{{{%
	\tableofcontents
	\newpage
	\setcounter{page}{1}
	\newpage
	%%%%%%%%%%%%%%%%%%%%%%%%%%%%%%%%%}}}%
	% Chapters                          %
	%%%%%%%%%%%%%%%%%%%%%%%%%%%%%%%%%{{{%
	\input{"1.design_flow"}
	\newpage
	\clearpage
	%\onehalfspacing % Stelle 1.5er Abstand ein
	%\setstretch{1.1} 
	\input{"2.basic_knowledge"}
	\newpage
	\clearpage
	
	%%%%%%%%%%%%%%%%%%%%%%%%%%%%%%%%%}}}%
	% Appendix                          %
	%%%%%%%%%%%%%%%%%%%%%%%%%%%%%%%%%{{{%
	
	%clear headers
	\fancyhead{}
	\fancyfoot{}
	\fancyfoot[CO, CE] {\thepage}
	
	\section{Appendix}
	
	%%%%%%%%%%%%%%%% >INSERT YOUR APPENDIX HERE>
	%\input{"7.appendix.tex"}
	%\newpage
	%\clearpage
	%\addcontentsline{toc}{section}{Appendix}
	%List of Figures
	\vspace{-20pt}
	\begingroup
	\addcontentsline{toc}{subsection}{List of Figures}
	\setlength{\cftparskip}{10pt}
	\listoffigures 
	\endgroup
	\newpage
	\clearpage
	%List of Tables
	\begingroup
	\addcontentsline{toc}{subsection}{List of Tables}
	\setlength{\cftparskip}{10pt}
	\listoftables
	\endgroup
	\newpage
	\clearpage
	%List of Listings
	%\renewcommand{\lstlistlistingname}{Verzeichnis der Quellcodes}
	\begingroup
	\addcontentsline{toc}{subsection}{List of Listings}
	\setlength{\itemsep}{20pt}
	\setlength{\parskip}{10pt}
	\renewcommand{\listlistingname}{List of Listings}
	\listoflistings 
	\endgroup
	\newpage
	\clearpage
	%List of Abbreviations
	\begingroup
	\addcontentsline{toc}{subsection}{List of Abbreviations}
	\renewcommand{\nomname}{List of Abbreviations}
	\renewcommand{\nompreamble}{\vspace{10pt}}
	%\setlength{\nomitemsep}{8pt}
	\printnomenclature[2cm]
	\endgroup
	\newpage
	\clearpage
	%Literatur:
	\addcontentsline{toc}{subsection}{References}
	\bibliographystyle{unsrt}
	\bibliography{myrefs}
	%%%%%%%%%%%%%%%%%%%%%%%%%%%%%%%%%}}}%
\end{document}